\begin{titlepage}
    \begin{center}
        
        \vspace*{1.6cm}
        
        \LARGE{ \bfseries{ \inserttitle } }\par
        
        \vspace{2.5cm}
        \includegraphics[width=0.25\textwidth]{crest}
        \vspace{2.5cm}
        
        \Large
        {\insertauthor}
        \large
        \\University of St Andrews
    
        \vspace{0.8cm}

        Supervisor\\
        \Large 
        \insertsupervisor

        \vspace{0.8cm}
        
        \large
        

        \vspace{0.8cm}
        A dissertation submitted for the degree of
        \\ \textit{\insertdegree}\\
        \insertdate

        \vfill
        
    \end{center}
\end{titlepage}